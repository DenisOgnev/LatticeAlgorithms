\newpage

\section{Нахождение ЭНФ (TODO)}

Будет разобрано два алгоритма - общий и алгоритм для матриц полного ранга строки, который используется в общем алгоритме.

\subsection{Алгоритм для матриц полного ранга строки}

Дана матрица $ \mathbf{B} \in \mathbb{Z}^{m \times n} $. Предположим, что у нас есть процедура AddColumn, которая работает за полиномиальное время и принимает на вход квадратную невырожденную ЭНФ матрицы $ \mathbf{H} \in \mathbb{Z}^{m \times m} $ и вектор $ \mathbf{b} $, а возвращает ЭНФ матрицы $ [\mathbf{H|b}] $. ЭНФ от $ \mathbf{B} $ может быть вычислена следующим образом:
\begin{enumerate}
\item Применить алгоритм Грама-Шмидта к столбцам $ \mathbf{B} $, чтобы найти $ m $ линейно независимых столбцов. Пусть $ \mathbf{B}’ $ - матрица размера $ m \times m $, заданная этими столбцами.
\item Вычислить $ d=\mathrm{det}(\mathbf{B}’) $, используя алгоритм Грама-Шмидта или любую другую процедуру с полиномиальным временем. Пусть $ \mathbf{H}_0=d \cdot \mathbf{I} $ будет диагональной матрицей с $ d $ на диагонали.
\item Для $ i=1,\ldots,n $ пусть $ \mathbf{H}_i $ это результат применения AddColumn ко входу $  \mathbf{H}_{i-1}\ $ и $ \mathbf{b}_i $.
\item Вернуть $ \mathbf{H}_n $.
\end{enumerate}

Разберем подпункты:

\begin{enumerate}
\item Необходимо найти линейно независимые столбцы матрицы. Их количество всегда будет равно $ m $, т.к. наша матрица полного ранга строки, а значит матрица, состоящая из этих столбцов, будет размера $ m \times m $. Для нахождения этих строк можно использовать алгоритм ортогонализации Грама-Шмитда: если $ \mathbf{b}_i^\ast=0 $, то $ i $-ая строка является линейной комбинацией других строк, и ее необходимо удалить. Реализация данного алгоритма находится в пространстве имен Utils в функции get\_linearly\_independent\_columns\_by\_gram\_schmidt. Полученная матрица будет названа $  \mathbf{B}’ $.
\item Для вычисления $ \mathrm{det} $ напишем функцию det\_by\_gram\_schmidt, которая принимает на вход матрицу и вычисляет $ \mathrm{det} $ по формуле $ d=\prod_i{\|\mathbf{b}_i^\ast\|} $ - сумма произведений длин всех элементов, полученных после применения ортогонализации Грама-Шмидта. Матрица $ \mathbf{H}_\mathbf{0} $ будет единичной матрицей размера $ m \times m $, умноженной на определитель. В результате все диагональные элементы будут равны $ d $.
\item Применяем функцию AddColumn (реализация находится в функции add\_column) к $ \mathbf{H}_0 $ и первому столбцу матрицы $ \mathbf{B} $ -- $ \mathbf{b}_0 $, получаем $ \mathbf{H}_1 $, повторяем для всех столбцов, получаем $ \mathbf{H}_n $.
\item $ \mathbf{H}_n $ является ЭНФ($ \mathbf{B} $).
\end{enumerate}

Алгоритм AddColumn на вход принимает квадратную невырожденную ЭНФ матрицы $ \mathbf{H} \in \mathbb{Z}^{m \times m} $ и вектор $ \mathbf{b} \in \mathbb{Z}^m $ и работает следующим образом. Если $ m = 0 $, то тут ничего не надо делать, и мы можем сразу вернуть $ \mathbf{H} $. В противном случае, пусть $ \mathbf{H} = $ и дальше:

\clearpage