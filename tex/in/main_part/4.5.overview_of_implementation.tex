\newpage

\section{Обзор программной реализации}

В ходе выполнения выпускной квалификационной работы была получена реализация описанных алгоритмов на языке C++. Для хранения исходного кода используется система контроля версий Git и сервис Github, где был создан репозиторий \cite{Repository}. Программная реализация должна использоваться как подключаемая библиотека. Структура проекта следующая:
\begin{itemize}

\item В папке src содержатся файлы с исходным кодом в формате .cpp.

\item В папке include содержатся подключаемые header файлы .hpp.

\item В папке tex содержатся исходные .tex файлы документа выпускной квалификационной работы.

\item В папке docs содержатся отчеты прошлых семестров.

\item В папке 3rdparty содержатся модули Git.

\item В папке cmake содержатся файлы для подключения сборок некоторых библиотек через CMake.

\item CMakeLists.txt --- файл CMake, использующийся для сборки проекта.

\end{itemize}

Проект автоматически собирается с помощью системы сборки CMake. Информация по сборке описана в README репозитория. По умолчанию отключена сборка документа выпускной квалификационной работы.

Программная реализация тестировалась с использованием компилятора G++ версии 6.3.0 в режиме сборки Release на ПК со следующими характеристиками: CPU: Intel(R) Core (TM) i5-9600KF CPU @ 3.70GHz, ОЗУ: DDR4, 16 ГБ (двухканальных режим 8х2), 2666 МГц. Тестирование проводилось на одинаковых данных.

\subsection{Вспомогательные функции}

Вспомогательные функции находятся в файле utils.cpp в пространстве имен Utils:

add\_column(HNF, column) $ \rightarrow $ $ \left[\text{HNF} | \text{column}\right] $ --- функция, используемая при нахождении ЭНФ. Принимает ЭНФ и возвращает $ \left[\text{HNF} | \text{column}\right] $.

reduce(vector, matrix) $ \rightarrow $ reduced\_vector --- функция для сокращения вектора относительно диагональных элементов входного базиса.

generate\_random\_matrix\_with\_full\_row\_rank(m, n, lowest, highest) $ \rightarrow $ matrix --- возвращает произвольную матрицу заданного размера с полным рангом строки с числами в заданном диапазоне.

generate\_random\_matrix(m, n, lowest, highest) $ \rightarrow $ matrix --- возвращает произвольную матрицу заданного размера с числами в заданном диапазоне.

get\_linearly\_independent\_columns\_by\_gram\_schmidt(matrix) $ \rightarrow $ result\_matrix --- возвращает линейно независимые столбцы матрицы и ортогонализованный базис.

get\_linearly\_independent\_rows\_by\_gram\_schmidt(matrix) $ \rightarrow $ result\_matrix --- возвращает линейно независимые строки матрицы, их индексы в исходной матрице, индексы удаленных строк и матрицу $ \mathbf{T} $.

gcd\_extended(a, b) $ \rightarrow $ g, x, y --- расширенный НОД алгоритм, возвращает g, x, y такие, что $ \mathrm{g} = \mathrm{xa} + \mathrm{yb} $.

add\_column\_GMP(HNF, column) $ \rightarrow $ $ \left[\text{HNF} | \text{column}\right] $ --- функция, используемая при нахождении ЭНФ. Принимает ЭНФ и возвращает $ \left[\text{HNF} | \text{column}\right] $. Использует реализацию больших чисел от GMP.

reduce\_GMP(vector, matrix) $ \rightarrow $ reduced\_vector --- функция для сокращения вектора относительно диагональных элементов входного базиса. Использует реализацию больших чисел от GMP.

generate\_random\_matrix\_with\_full\_row\_rank\_GMP(m, n, lowest, highest) $ \rightarrow $ matrix --- возвращает произвольную матрицу заданного размера с полным рангом строки с числами в заданном диапазоне. Использует реализацию больших чисел от GMP.

generate\_random\_matrix\_GMP(m, n, lowest, highest) $ \rightarrow $ matrix --- возвращает произвольную матрицу заданного размера с числами в заданном диапазоне. Использует реализацию больших чисел от GMP.

get\_linearly\_independent\_columns\_by\_gram\_schmidt\_GMP(matrix) $ \rightarrow $ result\_matrix --- возвращает линейно независимые столбцы матрицы и ортогонализованный базис. Использует реализацию больших чисел от GMP.

get\_linearly\_independent\_rows\_by\_gram\_schmidt\_GMP(matrix) $ \rightarrow $ result\_matrix --- возвращает линейно независимые строки матрицы, их индексы в исходной матрице, индексы удаленных строк и матрицу $ \mathbf{T} $. Использует реализацию больших чисел от GMP.

gcd\_extended\_GMP(a, b) $ \rightarrow $ g, x, y --- расширенный НОД алгоритм, возвращает g, x, y такие, что $ \mathrm{g} = \mathrm{xa} + \mathrm{yb} $.  Использует реализацию больших чисел от GMP.

generate\_random\_matrix\_with\_full\_column\_rank(m, n, lowest, highest) $ \rightarrow $ matrix --- возвращает произвольную матрицу заданного размера с полным рангом столбца с числами в заданном диапазоне.

generate\_random\_vector(m, lowest, highest) $ \rightarrow $ vector --- возвращает случайный вектор заданного размера с числами в заданном диапазоне.

projection(matrix, vector) $ \rightarrow $ result\_vector --- возвращает $ \text{vector} \perp \text{matrix} $.

closest\_vector(matrix, vector) $ \rightarrow $ result\_vector --- принимает набор векторов и целевой вектор, возвращает вектор из набора, ближайший к целевому.


\subsection{Ортогонализация Грама-Шмидта}

Реализация находится в файле algorithms.cpp в пространстве имен Utils и содержит 2 функции:

\begin{enumerate}

\item gram\_schmidt\_sequential(matrix, delete\_zero\_rows) $ \rightarrow $ result\_GS --- принимает на вход матрицу и флаг, указывающий, следует ли удалять нулевые строки, и возвращает ортогонализацию Грама-Шмидта.

\item gram\_schmidt\_parallel(matrix, delete\_zero\_rows) $ \rightarrow $ result\_GS --- принимает на вход матрицу и флаг, указывающий, следует ли удалять нулевые строки, и возвращает ортогонализацию Грама-Шмидта, вычисленную параллельным путем.

\end{enumerate}

\begin{table}[H]
  \caption{Время нахождения ортогонализации Грама-Шмидта}
  \centering
  \begin{tabular}{ | l | l | l | l | l | l | l | }
  \hline
  m & 50 & 200 & 600 & 1000 & 2500 & 5000 \\ \hline
  n & 50 & 200 & 600 & 1000 & 2500 & 5000  \\ \hline
  Время, сек & 0.001 & 0.013 & 0.35 & 1.57 & 24.1 & 191.7 \\ \hline
  \end{tabular}
  \label{table:GS_SEQ}
\end{table}

\begin{table}[H]
  \caption{Время параллельного нахождения ортогонализации Грама-Шмидта}
  \centering
  \begin{tabular}{ | l | l | l | l | l | l | l | }
  \hline
  m & 50 & 200 & 600 & 1000 & 2500 & 5000 \\ \hline
  n & 50 & 200 & 600 & 1000 & 2500 & 5000  \\ \hline
  Время, сек & 0.002 & 0.02 & 0.28 & 1.5 & 12.3 & 85.4 \\ \hline
  \end{tabular}
  \label{table:GS_PAR}
\end{table}

\subsection{Нахождение ЭНФ}

В ходе работы была получена реализация с использованием библиотеки Boost.Multiprecision. Реализация находится в файле algorithms.cpp в пространстве имен Algorithms::HNF и состоит из 4 функций:

\begin{enumerate}

\item HNF\_full\_row\_rank(matrix) $\rightarrow$ result\_HNF --- принимает на вход матрицу с полным рангом строки и возвращает ее ЭНФ. Использует встроенную реализацию больших чисел Boost.Multiprecision.

\item HNF(matrix) $\rightarrow$ result\_HNF --- принимает на вход матрицу и возвращает ее ЭНФ. Использует встроенную реализацию больших чисел Boost.Multiprecision.

\item HNF\_full\_row\_rank\_GMP(matrix) $\rightarrow$ result\_HNF --- принимает на вход матрицу с полным рангом строки и возвращает ее ЭНФ. Использует реализацию больших чисел от GMP.

\item HNF\_GMP(matrix) $\rightarrow$ result\_HNF --- принимает на вход матрицу и возвращает ее ЭНФ. Использует реализацию больших чисел от GMP.

\end{enumerate}

\begin{table}[H]
  \caption{Время работы ЭНФ}
  \centering
  \begin{tabular}{ | l | l | l | l | l | l | l | l | l | l | l | }
  \hline
  m & 5 & 10 & 17 & 25 & 35 & 50 & 75 & 100 & 100 & 125 \\ \hline
  n & 5 & 10 & 17 & 25 & 35 & 50 & 75 & 100 & 125 & 100 \\ \hline
  Время, сек & 0.001 & 0.005 & 0.05 & 0.24 & 1.03 & 4.27 & 23.2 & 78.3 & 117.1 & 104.7 \\ \hline
  \end{tabular}
  \label{table:HNF}
\end{table}

\begin{table}[H]
  \caption{Время работы ЭНФ с использованием GMP}
  \centering
  \begin{tabular}{ | l | l | l | l | l | l | l | l | l | l | l | }
  \hline
  m & 5 & 10 & 17 & 25 & 35 & 50 & 75 & 100 & 100 & 125 \\ \hline
  n & 5 & 10 & 17 & 25 & 35 & 50 & 75 & 100 & 125 & 100 \\ \hline
  Время, сек & 0.002 & 0.01 & 0.06 & 0.22 & 0.85 & 3.35 & 17.9 & 59.6 & 84.2 & 71.23 \\ \hline
  \end{tabular}
  \label{table:HNF_GMP}
\end{table}

По временам видно, что чем больше размер входной матрицы, тем сильнее идет замедление по времени. На матрицах больших размеров следует использовать реализацию, которая использует библиотеку GMP.


\subsection{Решение ПБВ}

Реализация находится в файле algorithms.cpp в пространстве имен Algorithms::CVP и состоит из 4 функций:

\begin{enumerate}

\item greedy\_recursive(matrix, vector) $ \rightarrow $ vector --- рекурсивный Greedy алгоритм, принимает на вход базис решетки и целевой вектор, возвращает вектор решетки, примерно ближайший к целевому.

\item greedy(matrix, vector) $ \rightarrow $ vector --- последовательный Greedy алгоритм, принимает на вход базис решетки и целевой вектор, возвращает вектор решетки, примерно ближайший к целевому.

\item branch\_and\_bound(matrix, vector) $ \rightarrow $ vector --- рекурсивный Branch and Bound алгоритм, принимает на вход базис решетки и целевой вектор, возвращает вектор решетки, ближайший к целевому.

\item greedy\_recursive(matrix, vector) $ \rightarrow $ vector --- параллельный рекурсивный Branch and Bound алгоритм, принимает на вход базис решетки и целевой вектор, возвращает вектор решетки, ближайший к целевому.

\end{enumerate}


\begin{table}[H]
  \caption{Время работы рекурсивного Greedy}
  \centering
  \begin{tabular}{ | l | l | l | l | l | l | l | l | l | l | l | l | l | }
  \hline
  m & 12 & 20 & 50 & 100 & 150 & 250 & 500 & 1000 & 1500 & 2500 & 3500 & 5000 \\ \hline
  n & 12 & 20 & 50 & 100 & 150 & 250 & 500 & 1000 & 1500 & 2500 & 3500 & 5000 \\ \hline
  Время, сек & 0.002 & 0.003 & 0.004 & 0.006 & 0.1 & 0.027 & 0.2 & 0.9 & 2.9 & 13.4 & 29.2 & 78.8 \\ \hline
  \end{tabular}
  \label{table:Greedy_recursive}
\end{table}

\begin{table}[H]
  \caption{Время работы нерекурсивного Greedy}
  \centering
  \begin{tabular}{ | l | l | l | l | l | l | l | l | l | l | l | l | l | }
  \hline
  m & 12 & 20 & 50 & 100 & 150 & 250 & 500 & 1000 & 1500 & 2500 & 3500 & 5000 \\ \hline
  n & 12 & 20 & 50 & 100 & 150 & 250 & 500 & 1000 & 1500 & 2500 & 3500 & 5000 \\ \hline
  Время, сек & 0.002 & 0.003 & 0.004 & 0.007 & 0.01 & 0.027 & 0.2 & 0.9 & 2.9 & 13.2 & 29 & 78.6 \\ \hline
  \end{tabular}
  \label{table:Greedy}
\end{table}

\begin{table}[H]
  \caption{Время работы нерекурсивного Greedy}
  \centering
  \begin{tabular}{ | l | l | l | l | l | l | }
  \hline
  m & 3 & 7 & 9 & 11 & 15 \\ \hline
  n & 3 & 7 & 9 & 11 & 11 \\ \hline
  Время, сек & 0.002 & 0.061 & 1.65 & 9.4 & 20.2 \\ \hline
  \end{tabular}
  \label{table:BB}
\end{table}

\begin{table}[H]
  \caption{Время работы параллельного Branch and Bound}
  \centering
  \begin{tabular}{ | l | l | l | l | l | l | l | }
  \hline
  m & 3 & 7 & 9 & 11 & 12 & 13 \\ \hline
  n & 3 & 7 & 9 & 11 & 12 & 13 \\ \hline
  Время, сек & 0.001 & 0.01 & 0.2 & 1.6 & 16.1 & 91.2 \\ \hline
  \end{tabular}
  \label{table:BB_parallel}
\end{table}

По временам видна заметная разница в скорости выполнения алгоритмов. Можно заметить, что сложность точного вычисления ПБВ сильно растет с увеличением количества столбцов базиса.

\clearpage