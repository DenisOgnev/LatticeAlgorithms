\newpage

\section{Основные определения (TODO)}

Матрица – прямоугольная таблица чисел, состоящая из $ n $ столбцов и $ m $ строк. Обозначается полужирной заглавной буквой, а ее элементы - строчными с двумя индексами (строка и столбец). При программировании использовалась стандартная структура хранения матриц:

$ \mathbf{A} = \left( 
\begin{array}{cccc}
a_{11} & a_{12} & \cdots & a_{1n} \\
a_{21} & a_{22} & \cdots & a_{2n} \\
\vdots & \vdots & \ddots & \vdots \\
a_{m1} & a_{m2} & \cdots & a_{mn}
\end{array} \right) $

Квадратная матрица – матрица, у которой число строк равно числу столбцов $ m = n $.

Единичная матрица – матрица, у которой диагональные элементы $ (i = j) $ равны единице.

Невырожденная матрица – квадратная матрица, определитель которой отличен от нуля.

Вектор – если матрица состоит из одного столбца $ (n = 1) $, то она называется вектором-столбцом. Если матрица состоит из одной строки $ (m = 1) $, то она называется вектором-строкой. Матрицы можно обозначать через вектора-столбцы и через вектора-строки: $ \mathbf{A} = \left[ \begin{array}{cccc}
\mathbf{a}_1 & \ldots & \mathbf{a}_n 
\end{array} \right] = \left[ \begin{array}{cccc}
\mathbf{a}^\mathbf{T}_1 \\
\vdots \\
\mathbf{a}^\mathbf{T}_m
\end{array} \right] $.

Линейная зависимость и независимость – пусть имеется несколько векторов одной размерности $ \mathbf{x}_1, \mathbf{x}_2, \ldots ,\mathbf{x}_k $ и столько же чисел $ \alpha_1, \alpha_2, \cdots, \alpha_k $. Вектор $ \mathbf{y}=\alpha_1 \mathbf{x}_1 + \alpha_2 \mathbf{x}_2 + \ldots + \alpha_k \mathbf{x}_k $ называется линейной комбинацией векторов $ \mathbf{x}_k $. Если существуют такие числа $ \alpha_i, i=1, \ldots, k $, не все равные нулю, такие, что $ \mathbf{y}=\mathbf{0} $, то такой набор векторов называется линейно зависимым. В противном случае векторы называются линейно независимыми.

Ранг матрицы – максимальное число линейно независимых векторов. Матрица называется матрицей с полным рангом строки, когда все строки матрицы линейно независимы. Матрица называется матрицей с полным рангом столбца, когда все столбцы матрицы линейно независимы.

Решетка - пусть $ \mathbf{B} = [\mathbf{b}_1, \ldots, \mathbf{b}_n] \in \mathbb{R}^{d \times n} $ - линейно независимые вектора из $ \mathbb{R}^d $. Решетка, генерируемая от $\mathbf{B}$ есть множество $$ \mathcal{L}(\mathbf{B}) = \lbrace \mathbf{Bx}: \mathbf{x} \in \mathbb{Z}^n \rbrace = \left\{ \sum\limits_{i=1}^n x_i \cdot \mathbf{b}_i: \forall i \ x_i \in \mathbb{Z} \right\} $$
всех целочисленных линейных комбинаций столбцов матрицы $\mathbf{B}$. Матрица $\mathbf{B}$ называется базисом для решетки $\mathcal{L}(\mathbf{B})$. Число $n$ называется рангом решетки. Если $n = d$, то решетка $\mathcal{L}(\mathbf{B})$ называется решеткой полного ранга или полноразмерной решеткой в $\mathbb{R}^d$.

Эрмитова нормальная форма - невырожденная матрица $ \mathbf{B}=\left[\mathbf{b}_1, \ldots, \mathbf{b}_n\right] \in \mathbb{R}^{m \times n}\ $ является Эрмитовой нормальной формой, если

\begin{itemize}
\item Существует $ 1 \le i_1 < \ldots < i_h \le m $ такое, что $ b_{i,j} \neq 0 \Rightarrow (j < h) \land (i \geq i_j) $ (строго убывающая высота столбца).
\item Для всех $ k>j, 0 \le b_{{i_j,k}}<b_{i_j,j} $, т.е. все элементы в строках $ i_j $ приведены по модулю $ b_{i_j, j} $.
\end{itemize}


\clearpage