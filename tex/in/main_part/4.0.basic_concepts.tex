\newpage

\section{Основные определения (TODO)}

Решетка. Пусть $ \mathbf{B} = [\mathbf{b}_1, ..., \mathbf{b}_n] \in \mathbb{R}^{d \times n} $ - линейно независимые вектора из $ \mathbb{R}^d $. Решетка, генерируемая от $\mathbf{B}$ есть набор $$ \mathcal{L}(\mathbf{B}) = \lbrace \mathbf{Bx}: \mathbf{x} \in \mathbb{Z}^n \rbrace = \left\{ \sum\limits_{i=1}^n x_i \cdot \mathbf{b}_i: \forall i \ x_i \in \mathbb{Z} \right\} $$
всех целочисленных линейных комбинаций столбцов матрицы $\mathbf{B}$. Матрица $\mathbf{B}$ называется базисом для решетки $\mathcal{L}(\mathbf{B})$. Число $n$ называется рангом решетки. Если $n = d$, то решетка $\mathcal{L}(\mathbf{B})$ называется решеткой полного ранга или полноразмерной решеткой в $\mathbb{R}^d$.

\clearpage