\newpage

\section{Обзор инструментов (TODO)}

Для программной реализации был выбран язык C++. Приоритет этому языку был отдан из-за его скорости, статической типизации, большому количеству написанных библиотек и богатой стандартной библиотеке. Сборка проекта осуществляется с помощью системы сборки CMake, при сборке можно указать флаги \verb!BUILD_DOCS! -- для сборки документа выпускной квалификационной работы, написанной в формате \LaTeX, \verb!BUILD_PARALLEL_BB! -- для сборки параллельной реализации алгоритма Branch and Bound и \verb!BUILD_GMP! -- для использования GMP. Для работы с матрицами была выбрана библиотека Eigen, для работы с большими числами используется часть библиотеки Boost -- Boost.Multiprecision, которая подключается в режиме Standalone. Используется встроенная в Boost реализация больших чисел и реализация от GMP.

Используется система контроля версий Git и сервис Github, все исходные файлы проекта доступны в онлайн репозитории. Для подключения Boost.Multiprecision используются модули Git.

\subsection{Обзор библиотеки Eigen}

Eigen - библиотека для работы с линейной алгеброй. Предоставляет шаблонные классы и методы для работы с матрицами, векторами и связанными алгоритмами. Является header-only библиотекой и не требует отдельной компиляции. Для работы не требует других библиотек, кроме стандратной.

Все необходимые классы находятся в заголовочном файле \verb!Eigen/Dense! и подключаются командой \verb!#include <Eigen/Dense>!. Для их использования необходимо указывать пространство имен \verb!Eigen!, например \verb!Eigen::Matrix2d!.

Используемые классы:

\verb!Matrix<typename Scalar, int RowsAtCompileTime, int ColsAtCompileTime>! -- шаблонный класс матрицы. Первый параметр шаблона отвечает за тип элементов матрицы, второй параметр за количество строк, третий за количество столбцов. Если количество строк/столбцов неизвестно на этапе компиляции, а будет найдено в процессе выполнения программы, то необходимо ставить количество строк/столбцов равным \verb!Eigen::Dynamic!, либо \verb!-1!. Имеет псевдонимы для различных встроенных типов (int, double, float) и размеров матриц (2, 3, 4), например \verb!Matrix3d! -- матрица элементов \verb!double! размера \verb!3x3!. 

\verb!Vector! и \verb!RowVector! -- вектор-столбец и вектора-строка, являются псевдонимами класса матриц, в которых количество строк или столбцов равно единице соответственно. Используются псевдонимы для различных встроенных типов (int, float, double) и размеров векторов (2, 3, 4), например \verb!Vector2f! -- вектор, состоящий из элементов \verb!float! размера \verb!3!.

С матрицами и векторами можно производить различные арифметические действия, например складывать и вычитать между собой, умножать и делить между собой и на скаляр. Все действия должны осуществляться по правилам линейной алгебры. 

Используемые методы:

\verb!matrix.rows()! -- получение количества строк.

\verb!matrix.cols()! -- получение количества столбцов.

\verb!vector.norm()! -- длина вектора.

\verb!vector.squaredNorm()! -- квадрат длины вектора.

\verb!matrix << elems! -- comma-инициализация матрицы, можно вставлять скалярные типы, матрицы, вектора.

\verb!Eigen::MatrixXd::Identity(m, m)! -- получение единичной матрицы размера $ m \times m $.

\verb!Eigen::VectorXd::Zero(m)! -- получение нулевого вектора размера $ m $.

\verb!matrix.row(index)! -- получение строки матрицы по индексу.

\verb!matrix.col(index)! -- получение столбца матрицы по индексу.

\verb!matrix.row(index) = vector! -- установить строку матрицы значениями вектора.

\verb!matrix.col(index) = vector! -- установить столбец матрицы значениями вектора.

\verb!matrix.block(startRow, startCol, endRow, endCol)! -- получение подматрицы по индексам.

\verb!matrix.block(startRow, startCol, endRow, endCol) = elem! -- установка блока матрицы по индексам значением elem.

\verb!matrix.cast<type>()! -- привести матрицу к типу type.

\verb!vector1.dot(vector2)! -- скалярное произведение двух векторов.

\verb!vector.tail(size)! -- получить с конца вектора size элементов.

\verb!matrix(i, j)! -- получение элемента матрицы по индексам.

\verb!vector(i)! -- получение элемента вектора по индексу.

\verb!matrix(i, j) = elem! -- установка элемента матрицы по индексам значением elem.

\verb!vector(i) = elem! -- установка элемента вектора по индексу значением elem.

\verb!for (const Eigen::VectorXd &vector : matrix.colwise())! -- цикл по столбцам матрицы.

\verb!for (const Eigen::VectorXd &vector : matrix.rowwise())! -- цикл по строкам матрицы.

\subsection{Обзор библиотеки Boost.Multiprecision}

Boost.Multiprecision -- часть библиотеки Boost, подключается в режиме Standalone и не требует подключения основной библиотеки, что позволяет не использовать модули, которые не требуются и уменьшить итоговый размер. Все классы находятся в пространстве имен \verb!boost::!\\ \verb!multiprecision!. Для подключения используется директива препоцессора \verb!#include <boost/! \\ \verb!multiprecision/cpp_!тип\verb!.hpp>!. Если при сборке CMake будет указан флаг \verb!BUILD_GMP=ON!, то будет использована обертка от Boost над библиотекой GMP. Классы, связанные с GMP, подключаются с помощью \verb!#include <boost/multiprecision/gmp.hpp>!. В документации Boost указано, что реализация GMP работает быстрее.

Библиотека предоставляет классы для работы с целыми, рациональными числами и числами с плавающей запятой неограниченной точности. Размер этих чисел ограничен только количеством оперативной памяти. 

Используемые классы:

\verb!cpp_int! -- класс целых чисел.

\verb!cpp_rational! -- класс рациональных чисел.

\verb!cpp_bin_float_double! -- класс чисел с плавающей запятой с увеличенной точностью.

\verb!mpz_int! -- класс целых чисел, использующий реализацию GMP.

\verb!mpq_rational! -- класс рациональных чисел, использующий реализацию GMP.

\verb!mpf_float_50! -- класс чисел с плавающей запятой, использующий реализацию GMP.

Используемые методы:

\verb!sqrt(int)! -- квадратный корень из целого числа.

\verb!numerator(rational)! -- числитель рационального числа.

\verb!denominator(rational)! -- знаменатель рационального числа.

\clearpage