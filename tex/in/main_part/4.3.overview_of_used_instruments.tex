\newpage

\section{Обзор инструментов (TODO)}

Для программной реализации был выбран язык C++. Приоритет этому яызку отдается из-за его скорости, статической типизации и большому количеству написанных библиотек. Сборка проекта осуществляется с помощью системы сборки CMake, при сборке она автоматически собирает документ выпускной квалификационной работы, написанный в формате \LaTeX. Для работы с матрицами была выбрана библиотека Eigen, для работы с большими числами используется часть библиотеки Boost Boost.Multiprecision, которая подключается в режиме Standalone.

Используется система контроля версий Git и сервис Github, все исходные файлы проекта доступны в онлайн репозитории. Для Boost.Multiprecision используются модули Git.

\subsection{Обзор библиотеки Eigen}

Eigen - шаблонная библиотека для работы с линейной алгеброй. Предоставляет классы и методы для работы с матрицами, векторами и связанными алгоритмами. Является header-only библиотекой, не требует отдельной компиляции и линковки. Для работы не требует других библиотек, кроме стандратной.

Все необходимые классы находятся в заголовочном файле \verb!Eigen/Dense! и подключается командой \verb!#include <Eigen/Dense>!. Все используемые классы находятся в пространстве имен Eigen.

Используемые классы:

\verb!Matrix<typename Scalar, int RowsAtCompileTime, int ColsAtCompileTime>! -- шаблонный класс матрицы. Первый параметр шаблона - тип элементов матрицы, второй параметр -- количество строк, третий -- количество столбцов. Если количество строк/столбцов неизвестно на стадии компиляции, а будет найдено в процессе выполнения программы, то необходимо ставить количество строк/столбцов равным \verb!Eigen::Dynamic!, либо \verb!-1!. Имеет псевдонимы для различных типов и размеров матриц, например \verb!Matrix3d! -- матрица элементов \verb!double! размера \verb!3x3!. 

\verb!Vector! и \verb!RowVector! -- псевдонимы класса матриц, в которых количество строк/столбцов равно единице. Используются псевдонимы для различных типов и размеров векторов, например \verb!Vector2f! -- вектор, состоящий из элементов \verb!float! размера \verb!3!.

Матрицы и вектора можно складывать и вычитать между собой, умножать и делить между собой и на скаляр. 

Используемые методы:

\verb!matrix.rows()! -- получение количества строк.

\verb!matrix.cols()! -- получение количества столбцов.

\verb!vector.norm()! -- длина вектора.

\verb!vector.squaredNorm()! -- квадрат длины вектора.

\verb!matrix << elems! -- comma-инициализация матрицы, можно вставлять скалярные типы, матрицы, вектора.

\verb!Eigen::MatrixXd::Identity(m, m)! -- получение единичной матрицы размера $ m \times m $.

\verb!Eigen::VectorXd::Zero(m)! -- получение нулевого вектора размера $ m $.

\verb!matrix.row(index)! -- получение строки матрицы по индексу.

\verb!matrix.col(index)! -- получение столбца матрицы по индексу.

\verb!matrix.row(index) = vector! -- установить строку матрицы значениями вектора.

\verb!matrix.col(index) = vector! -- установить столбец матрицы значениями вектора.

\verb!matrix.block(startRow, startCol, endRow, endCol)! -- получение подматрицы по индексам.

\verb!matrix.block(startRow, startCol, endRow, endCol) = elem! -- установка блока матрицы по индексам значением elem.

\verb!matrix.cast<type>()! -- привести матрицу к типу type.

\verb!vector1.dot(vector2)! -- скалярное произведение двух векторов.

\verb!vector.tail(size)! -- получить с конца вектора size элементов.

\verb!matrix(i, j)! -- получение элемента матрицы по индексам.

\verb!vector(i)! -- получение элемента вектора по индексу.

\verb!matrix(i, j) = elem! -- установка элемента матрицы по индексам значением elem.

\verb!vector(i) = elem! -- установка элемента вектора по индексу значением elem.

\verb!for (const Eigen::VectorXd &vector : matrix.colwise())! -- перебор матрицы по столбцам.

\verb!for (const Eigen::VectorXd &vector : matrix.rowwise())! -- перебор матрицы по строкам.

\subsection{Обзор библиотеки Boost.Multiprecision}

Multiprecision -- часть библиотеки Boost. Подключается в режиме Standalone и не требует подключения основной библиотеки. Все классы находятся в пространстве имен \\ \verb!boost::multiprecision!. Для подключения используется директива \\ \verb!#include <boost/multiprecision/cpp_!тип\verb!.hpp>!.

Библиотека предоставляет классы для работы с целыми, рациональными числами и числами с плавающей запятой, которые имеют большую точность, чем встроенные в C++ типы данных. Точность и размер чисел ограничен количеством оперативной памяти. 

Используемые классы:

\verb!cpp_int! -- класс целых чисел.

\verb!cpp_rational! -- класс рациональных чисел.

\verb!cpp_bin_float_double! -- класс чисел с плавающей запятой с увеличенной точностью.

Используемые методы:

\verb!sqrt(int)! -- квадратный корень из целого числа.

\verb!numerator(rational)! -- числитель рационального числа.

\verb!denominator(rational)! -- знаменатель рационального числа.

\clearpage