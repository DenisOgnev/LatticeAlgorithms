\newpage

\section{Решение ПБВ (TODO)}

Будет разобрано два алгоритма - жадный метод, работающий за полиномиальное время, но дающий приближенное решение, и метод ветвей и границ, работающий за суперполиномиальное время, но точно решающий проблему ближайшего вектора.

\subsection{Определение проблемы}

Рассмотрим проблему ближайшего вектора (ПБВ): Дан базис решетки $ \mathbf{B} \in \mathbb{R}^{d \times n} $ и вектор $ \mathbf{t} \in \mathbb{R}^d $, найти точку решетки $ \mathbf{Bx} $ $ (\mathbf{x} \in \mathbb{Z}^n) $ такую, что $ ||\mathbf{t - Bx}|| $ (расстояние от точки до решетки) минимально. Это задача оптимизации (минимизации) с допустимыми решениями, заданными всеми целочисленными векторами $ \mathbf{x} \in \mathbb{Z}^n $, и целевой функцией $ f(\mathbf{x}) = ||\mathbf{t - Bx}|| $.

Пусть $ \mathbf{B} = [\mathbf{B}', \mathrm{b}] $ и $ \mathbf{x} = (\mathbf{x}', x) $, где $ \mathbf{B}' \in \mathbb{R}^{d \times \left( n \mathrm{-} 1 \right)} $, $ \mathbf{b} \in \mathbb{R}^d $, $ \mathbf{x}' \in \mathbb{Z}^{n-1} $ и $ x \in \mathbb{Z} $.
Заметим, что если зафиксировать значение $ x $, то задача $ \text{ПБВ}(\mathbf{B, t}) $ потребует найти значение $ \mathbf{x}' \in \mathbb{Z}^{n - 1} $ такое, что 
$$ ||\mathbf{t} - (\mathbf{B}'\mathbf{x}' + \mathbf{b}x)|| = ||(\mathbf{t} - \mathbf{b}x)-\mathbf{B}'\mathbf{x}'|| $$ 
минимально. Это также экземпляр ПБВ $ (\mathbf{B}', \mathbf{t}') $ с измененным вектором $ \mathbf{t}' = \mathbf{t} - \mathbf{b}x$ , и решеткой меньшего размера $ \mathcal{L}(\mathbf{B}') $. В частности, пространство решений сейчас состоит из $ (n – 1) $ целочисленных переменных $ \mathbf{x}' $. Это говорит о том, что можно решить ПБВ путем установки значения $ \mathbf{x} $ по одной координате за раз.
Есть несколько способов превратить этот подход к уменьшению размерности в алгоритм, используя некоторые стандартные методы алгоритмического программирования. Простейшие методы:

\begin{enumerate}
\item Жадный метод, который выдает приближенные значения, но работает за полиномиальное время
\item Метод ветвей и границ, который выдает точное решение за суперэкспоненциальное время.
\end{enumerate}

Оба метода основаны на очень простой нижней оценке целевой функции:
$$ \min \limits_{x}f(\mathbf{x}) = dist\left(\mathbf{t}, \mathcal{L}\left(\mathbf{B}\right)\right)\geq dist \left(\mathbf{t}, span\left(\mathbf{B}\right)\right) = ||\mathbf{t} \perp \mathbf{B} || $$

\subsection{Жадный метод: алгоритм ближайшей плоскости Бабая}

Суть жадного метода состоит в выборе переменных, определяющих пространство решений, по одной, каждый раз выбирая значение, которые выглядит наиболее многообещающим. В нашем случае, выберем значение x, которое дает наименьшее возможное значение для нижней границы $ ||\mathbf{t}' \perp \mathbf{B}' || $. Напомним, что $ \mathbf{B}=\left[\mathbf{B}', \mathbf{b}\right] $ и $ \mathbf{x}=\left(\mathbf{x}' ,x\right) $, и что для любого фиксированного значения $ x $, ПБВ $ (\mathbf{B}, \mathbf{t}) $ сводится к ПБВ $ (\mathbf{B}',\mathbf{t}') $, где $ \mathbf{t}'=\mathbf{t}-\mathbf{b}x $. Используя $ ||\mathbf{t}' \perp \mathbf{B}' || $ для нижней границы, мы хотим выбрать значение $ x $ такое, что 
$$ || \mathbf{t}' \perp \mathbf{B}' || = || \mathbf{t} - \mathbf{b}x \perp \mathbf{B}' || = || (\mathbf{t} \perp \mathbf{B}') - (\mathbf{b} \perp \mathbf{B}')x || $$ 
как можно меньше. Это очень простая 1-размерная ПБВ проблема (с решеткой $ \mathcal{L}\left(\mathbf{b} \perp \mathbf{B}'\right) $ и целью $ \mathbf{t} \perp \mathbf{B}') $, которая может быть сразу решена установкой
$$ x = \left\lfloor \left\langle t,b^* \right\rangle \over ||b^*||^2 \right\rceil $$
где $ \mathbf{b}^* = \mathbf{b} \perp \mathbf{B}' $ компонента вектора $ \mathbf{b} $, ортогональная другим базисным векторам. Полный алгоритм приведен ниже: \newline
$ \text{Greedy}([], \mathbf{t}) = 0 $ \newline
$ \text{Greedy}([\mathbf{B},\mathbf{b}],\mathbf{t}) = c \cdot \mathbf{b} + \text{Greedy}(\mathbf{B},\mathbf{t} - c \cdot \mathbf{b}) $

$ \text{где } \mathbf{b}^* = \mathbf{b} \perp \mathbf{B} $

$ \qquad x = \left\langle \mathbf{t},\mathbf{b}^* \right\rangle / \left\langle \mathbf{b}^*,\mathbf{b}^* \right\rangle $

$ \qquad c = \left\lfloor x \right\rceil $.

\subsection{Пример жадного метода}

\subsection{Метод ветвей и границ}

Структура похожа на жадный алгоритм, но вместо жадной установки $ x_n $ на наиболее подходящее значение (то есть на то, для которого нижняя граница расстояния $ \mathbf{t}' \perp \mathbf{B}' $ минимальна), мы ограничиваем множество всех возможных значений для $ x $ , и затем мы переходим на каждую из них для решения каждой соответствующей подзадачи независимо. В заключении, мы выбираем наилучшее возможное решение среди возвращенных всеми ветками.

Чтобы ограничить значения, которые может принимать $ x $, нам также нужна верхняя граница расстояния от цели до решетки. Ее можно получить несколькими способами. Например, можно просто использовать $ ||\mathbf{t} || $ (расстояние от цели до начала координат) в качестве верхней границы. Но лучше использовать жадный алгоритм, чтобы найти приближенное решение $ \mathbf{v} = \text{Greedy}(\mathbf{B}, \mathbf{t}) $, и использовать $ || \mathbf{t} - \mathbf{v} || $ в качестве верхней границы. Как только верхняя граница $ u $  установлена, можно ограничить переменную $ x $ такими значениями, что $ (\mathbf{t} - x\mathbf{b}) \perp \mathbf{B}' || \leq u $.

Окончательный алгоритм похож на жадный метод и описан ниже: \newline
$ \text{Branch\&Bound}([], \mathbf{t}) = 0 $ \newline
$ \text{Branch\&Bound}([\mathbf{B}, \mathbf{b}], \mathbf{t}) = \text{closest}(V,\textbf{t})  $ 

$ \text{где } \mathbf{b}^* = \mathbf{b} \perp \mathbf{B} $ 

$ \qquad \mathbf{v} = \text{Greedy}(\mathbf{B},\mathbf{t}) $

$ \qquad X = {x: || (\mathbf{t} - x\mathbf{b}) \perp \mathbf{B} || \leq || \mathbf{t} - \mathbf{v} ||} $

$ \qquad V = {x \cdot \mathbf{b} + \text{Branch\&Bound}(\mathbf{B},\mathbf{t} - x \cdot \mathbf{b}):x \in X} $

где $ \text{closest}(V, \mathbf{t}) $ выбирает вектор в $ V \subset \mathcal{L}(\mathbf{B}) $ ближайший к цели $ \mathbf{t} $.  

Как и для жадного алгоритма, производительность метода Ветвей и Границ может быть произвольно плохой, если мы сперва не сократим базисы.

Сложность алгоритма заключается в нахождении множества $ X $. Его можно найти, используя выражение, выведенное в прошлом алгоритме: $ x = \frac{\left\langle t,b^* \right\rangle}{||b^*||^2} $. С помощью него мы найдем $ x $, который точно удовлетворяет множеству, а затем будет увеличивать/уменьшать до тех пор, пока выполняется условие $ || (\mathbf{t} - x\mathbf{b}) \perp \mathbf{B} || \leq || \mathbf{t} - \mathbf{v} || $.

\subsection{Пример метода ветвей и границ}

\subsection{Сложность алгоритмов}

\subsection{Обзор программной реализации}

\subsection{Применение}

\clearpage