\newpage

\section{Обзор инструментов}

Для программной реализации был выбран язык C++. Приоритет этому языку был отдан из-за его скорости, статической типизации, большому количеству написанных библиотек и обширной стандартной библиотеке. Сборка проекта осуществляется с помощью системы сборки CMake, при сборке можно указать флаги 
\begin{itemize}

\item BUILD\_DOCS --- для сборки документа выпускной квалификационной работы, написанной в формате Latex;

\item BUILD\_PARALLEL\_BB -- для сборки параллельной реализации алгоритма Branch and Bound;

\item BUILD\_GMP -- для использования библиотеки GMP.

\end{itemize} 

Для работы с матрицами была выбрана библиотека Eigen, для работы с большими числами используется часть библиотеки Boost --- Boost.Multiprecision, которая подключается в режиме Standalone. Используется встроенная в Boost реализация больших чисел и реализация от GMP.

Используется система контроля версий Git и сервис Github, все исходные файлы проекта доступны в онлайн репозитории. Для подключения Boost.Multiprecision используются модули Git.

\subsection{Обзор библиотеки Eigen}

Eigen - библиотека для работы с линейной алгеброй, предоставляет шаблонные классы для работы с матрицами и векторами. Является header-only библиотекой и не требует отдельной компиляции, для работы не требует других библиотек, кроме стандратной.

Все необходимые классы находятся в заголовочном файле Eigen/Dense и подключаются директивой \#include <Eigen/Dense>, для их использования необходимо указывать пространство имен Eigen, например Eigen::Matrix2d.

Используемые классы:

Matrix<typename Scalar, int RowsAtCompileTime, int ColsAtCompileTime> --- шаблонный класс матриц. Первый параметр шаблона отвечает за тип элементов матрицы, второй параметр за количество строк, третий за количество столбцов. Если количество строк/столбцов неизвестно на этапе компиляции, а будет найдено в процессе выполнения программы, то необходимо ставить количество строк/столбцов равным Eigen::Dynamic, либо -1. Имеет псевдонимы для различных встроенных типов (int, double, float) и размеров матриц (2, 3, 4), например Matrix3d --- матрица элементов double размера $ 3 \times 3 $. 

Vector и RowVector --- вектор-столбец и вектора-строка соответственно, являются псевдонимами класса матриц, в которых количество строк/столбцов равно единице. Используются псевдонимы для различных встроенных типов (int, float, double) и размеров векторов (2, 3, 4), например Vector2f --- вектор, состоящий из элементов float размера $ 3 $.

С матрицами и векторами можно производить различные арифметические действия, например складывать и вычитать между собой, умножать и делить между собой и на число. Все действия должны осуществляться по правилам линейной алгебры. 

Используемые методы:

matrix.rows() --- получение количества строк.

matrix.cols() --- получение количества столбцов.

vector.norm() --- длина вектора.

vector.squaredNorm() --- квадрат длины вектора.

matrix << elems --- comma-инициализация матрицы, можно инициализировать матрицу через скалярные типы, матрицы и векторы.

Eigen::MatrixXd::Identity(m, m) --- получение единичной матрицы размера $ m \times m $.

Eigen::VectorXd::Zero(m) --- получение нулевого вектора размера $ m $.

matrix.row(index) --- получение строки матрицы по индексу.

matrix.col(index) --- получение столбца матрицы по индексу.

matrix.row(index) = vector --- установить строку матрицы значениями вектора.

matrix.col(index) = vector --- установить столбец матрицы значениями вектора.

matrix.block(startRow, startCol, endRow, endCol) --- получение подматрицы по индексам.

matrix.block(startRow, startCol, endRow, endCol) = elem --- установка блока матрицы по индексам значением elem.

matrix.cast<type>() --- привести матрицу к типу type.

vector1.dot(vector2) --- скалярное произведение двух векторов.

vector.tail(size) --- получить с конца вектора size элементов.

matrix(i, j) --- получение элемента матрицы по индексам.

vector(i) --- получение элемента вектора по индексу.

matrix(i, j) = elem --- установка элемента матрицы по индексам значением elem.

vector(i) = elem --- установка элемента вектора по индексу значением elem.

for (const Eigen::VectorXd \&vector : matrix.colwise()) --- цикл по столбцам матрицы.

for (const Eigen::VectorXd \&vector : matrix.rowwise()) --- цикл по строкам матрицы.

\subsection{Обзор библиотеки Boost.Multiprecision}

Boost.Multiprecision --- часть библиотеки Boost, подключается в режиме Standalone, что позволяет не подключать основную библиотеку и не использовать модули, которые не требуются, в конечном итоге уменьшив итоговый размер программы. Все классы находятся в пространстве имен boost::multiprecision. Для подключения библиотеки используется директива \#include <boost::mul\-ti\-pre\-ci\-si\-on/cpp\_тип.hpp. Если при сборке CMake будет указан флаг BUILD\_GMP=ON, то будет использована обертка от Boost над библиотекой GMP. Классы, связанные с GMP, подключаются с помощью \#include <boost/mul\-ti\-pre\-ci\-si\-on/gmp.hpp>. В документации Boost сказано, что реализация GMP работает быстрее, что будет видно показано далее.

Библиотека предоставляет классы для работы с целыми, рациональными числами и числами с плавающей запятой неограниченной точности. Размер этих чисел ограничен только количеством оперативной памяти. 

Используемые классы:

cpp\_int --- класс целых чисел.

cpp\_rational --- класс рациональных чисел.

cpp\_bin\_float\_double --- класс чисел с плавающей запятой с увеличенной точностью.

mpz\_int --- класс целых чисел, использующий реализацию GMP.

mpq\_rational --- класс рациональных чисел, использующий реализацию GMP.

mpf\_float\_50 --- класс чисел с плавающей запятой, использующий реализацию GMP.

Используемые методы:

sqrt(int) --- квадратный корень из целого числа.

numerator(rational) --- числитель рационального числа.

denominator(rational) --- знаменатель рационального числа.

\clearpage