\newpage

%\addcontentsline{toc}{section}{Введение}
\section{Введение}

Криптография занимается разработкой методов преобразования (шифрования) информации с целью ее зашиты от незаконных пользователей. Самыми известными вычислительно трудными задачами считаются проблема вычисления дискретного логарифма и факторизация (разложение на множители) целых чисел. Для этих задач неизвестны эффективные (работающие за полиномиальное время) алгоритмы. С развитием квантовых компьютеров было показано существование полиномиальных алгоритмов решения задач дискретного логарифмирования и разложения числа на множители на квантовых вычислителях, что заставляет искать задачи, для которых неизвестны эффективные квантовые алгоритмы. В области постквантовой криптографии фаворитом считается криптография на решетках. Считается, что такая криптография устойчива к квантовым компьютерам.

Предметом исследования данной работы являются алгоритмы для нахождения Эрмитовой нормальной формы и решения проблемы ближайшего вектора. Целью работы является получение программной реализации алгоритмов для нахождения ЭНФ за полиномиальное время, приблизительного решения ПБВ за полиномиальное время и точного решения ПБВ за суперполиномиальное время. Необходимо будет показать, как можно использовать данные алгоритмы на практике. В качестве базы, откуда взяты теоретические основы и описание алгоритмов для программирования, будем использовать серию лекций по основам алгоритмов на решетках и их применении.


\clearpage

