\newpage

%\addcontentsline{toc}{section}{Введение}
\section{Введение}

Криптография --- наука, которая занимается методами преобразования (шифрования) с целью обеспечения конфиденциальности, целостности данных, аутентификации и зашиты информации от незаконных пользователей. Самыми известными вычислительно трудными задачами считаются проблема вычисления дискретного логарифма и факторизация (разложение на множители) целых чисел. Для этих задач неизвестны эффективные (работающие за полиномиальное время) алгоритмы. С развитием квантовых компьютеров было показано существование полиномиальных алгоритмов решения задач дискретного логарифмирования и разложения числа на множители на квантовых вычислителях\cite{Shor}, что заставляет искать задачи, для которых неизвестны эффективные квантовые алгоритмы. В области постквантовой криптографии фаворитом можно назвать криптографию на решетках, т.к считается, что она устойчива к квантовым компьютерам. Поэтому изучение задач теорий решеток является основной целью при построении устойчивых криптосистем на решетках.

Предметом исследования данной работы являются алгоритмы для нахождения Эрмитовой нормальной формы и решения проблемы ближайшего вектора. Целью работы является получение программной реализации алгоритмов для нахождения ЭНФ за полиномиальное время, приблизительного решения ПБВ за полиномиальное время и точного решения ПБВ за суперполиномиальное время. Необходимо будет показать, как можно использовать данные алгоритмы на практике. В качестве теоретической базы, откуда взяты основы и описание алгоритмов для программирования, была использована серия лекций по решеткам и решеточным алгоритмам.


\clearpage

